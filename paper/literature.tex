\section{Background/related work}
There have been several attempts on retrieving knowledge from ADHD subreddits.

\subsection{Topic/goal}
\citeauthor{scalzo2024using} goal is: ``to research and open a window to understand how adults identifying with ADHD talk about their lived experiences.''

\citeauthor{ginapp2023experiences}: ``This study aimed to use interpretive phenomenological analysis (IPA) to better understand how young adults with ADHD interpret their experiences interacting with society, managing interpersonal relationships, and building community.''

\citeauthor{eagle2023you}: ``How does the ADHD community leverage existing social media platforms to provide support previously contained within domain-specific OHCs?''

\subsection{Methods}
\citeauthor{scalzo2024using}: NLP
\citeauthor{ginapp2023experiences}: 
\citeauthor{eagle2023you}:

\subsection{Findings}
\citeauthor{scalzo2024using}:
In addition, users may share “life hack”-style advice, such as keeping multiple laundry hampers in different locations around the living space to address executive functioning issues which may otherwise lead to clothes being scattered on the floor. Notably, much of this advice encourages those with ADHD to relax or forgo societal norms around what a “clean” or “successful” living space looks like. Instead, readers are encouraged to do “what works for them.” These coping strategies are in-line with ones which a client-centered therapist operating from an evidence-based approach would recommend.

\citeauthor{ginapp2023experiences}: 
\citeauthor{eagle2023you}:
