\section{Introduction}
The diagnosis of Attention Deficit Hyperactivity Disorder (ADHD) has become more prevalent over the last few years. Increased awareness of the public is believed to play a big role \cite{abdelnour2022adhd}. This can be observed in the increased representation of those with ADHD on social media \cite{eagle2023you}.

A recent study has explored the emerging ADHD themes in social media posts. A total of 23 unique themes were identified. The study shortly addressed that some posts are related to sharing coping strategies. Although these strategies were not based on evidence, they seem to mirror the evidence based strategies \cite{scalzo2024using}.

Another study about adult experiences has found that: ``Many felt that strategies developed by people with ADHD were uniquely helpful'' \cite{ginapp2023experiences}.

The goal of this research project is to further analyse user generated coping strategies and gain insights into which themes they have and how prevalent, useful, etc. they seemingly are.

\begin{enumerate}
	\item Which different themes/topics are discussed in user generated coping strategies?
	\item How do people experience user generated coping strategies?
\end{enumerate}
